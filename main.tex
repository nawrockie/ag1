\documentclass[11pt]{article}
\usepackage{times}
\usepackage{verbatim}
\usepackage[pdftex]{graphicx}    
%\usepackage[pdftex]{color}  
\usepackage{fullpage}
\usepackage{url}
\usepackage[numbers,sort&compress]{natbib}
\usepackage{setspace}
\doublespacing
\setcounter{secnumdepth}{0}

\bibliographystyle{unsrtnat}

% \draftfalse: submission version. Legends,tables at end. No figs included.
% \drafttrue:   online preprint. Figures and table inline.
\newif\ifdraft
%\draftfalse
\drafttrue

\begin{document}



\title{Group I Introns in Archaea}
\author{Eric P. Nawrocki(1), Tom A. Jones(2), and Sean R. Eddy(2)\\
(1) National Center for Biotechnology Information \\
U.S. National Library of Medicine \\
Bethesda, MD 20894, USA \\
(2) Howard Hughes Medical Institute \\
Harvard University \\
16 Divinity Avenue \\
Cambridge, MA 02138, USA 
}
%\date{\today}
\maketitle


%%%%%%%%%%%%%%%%%%%%%%%%%%%%%%%%%%%%%%%%%%%%%%%%%%%%%%%%%%%%%%%%%%%%%%

%\begin{abstract}
%\end{abstract}

%%%%%%%%%%%%%%%%%%%%%%%%%%%%%%%%%%%%%%%%%%%%%%%%%%%%%%%%%%%%%%%%%%%%%%

\section{Introduction}

There are four types of introns classified based on their splicing
mechanism. Spliceosomal introns are found in nuclear messenger RNAs of
eukaryotes and are removed by the spliceosome, a ribonucleoprotein
complex. Group I introns are self-splicing\citep{Cech82}, adopt a
highly conserved tertiary structure crucial to their splicing ability,
and are found in bacteria, bacteriophages, fungal mitochondria,
chloroplasts and in protist rRNA genes.  Group II introns are also
self-splicing but via a different mechanism than group I introns which
resembles that of spliceosomal introns (REF). Group II introns are
found in bacteria, and mitochondrial and chloroplast genomes of fungi,
plants, protists and are rare in animals and archaea
\citep{Rest03}. Finally, tRNA introns are present in eukaryotes and
arcahea, where they are recognized and cleaved differently. In
eukaryotes, tRNA introns always occur at a particular position (the
anticodon loop) and are recogized and removed based on a mechanism
that relies on this position. In archaea, a bulge-helix-bulge (BHB)
structure is recognized by a tRNA endonuclease which cleaves the
intron. 

Group I introns typically range between 250 and 500 nucleotides in
length, and vary widely at the sequence level outside of a few core
domains (labelled P, Q, R, S)\citep{Hausner14}. The tertiary structure
is highly conserved, however, and includes up to 10 helices labelled
P1-P10. The highly conserved 'core' of the group I intron includes the
P3, P5, P6, P7 (a pseudoknot), P8, and P9.0 helices and the
nucleotides which connect them. Five types of group
I introns (IA, IB, IC, ID, and IE) have been defined based on
different configurations of the ten helices, conserved sequence
elements, and presence of additional structure features \citep{Michel90}.
All five types are present in the four phylogenetic groups in which
group I introns are found (bacteria, viruses, chloroplasts,
mitochondria, and eukaryotic nuclear genes) with the following
exceptions: IE is only found in eukaryotic nuclear genes and
mitochondria; ID has not yet been found in chloroplasts; and IB, ID,
and IE have not yet been found in viruses. These five types have been
further classified into 14 subtypes based mainly on specific sequences
of the P7 pseudoknot and other structural characteristics. The GISSD
database has created alignments and statistical profile models called
covariance models of the conserved sequence and secondary structure of
each of the 14 subtypes and used those models to predict more than
17,000 group I intron sequences, about 95\% of which belong to the IC1
group within the tRNA^{leu} chloroplast gene.

Many group I intron sequences include an open reading frame encoding a
homing endonuclease gene (HEG) which are responsible for their
mobility and consequently their existence in modern day genomes. HEGs
are sequence specific endonucleases which cleave double stranded DNA
and insert their own encoding sequences at specific recognition
sequences of length 12-40 nucleotdies (REF). There are at least four
types of HEGs, classified by their amino acid sequence motifs:
LADLIDADG, GIY-YIG, H-N-H, and His-Cys box.  Group Is and HEGs are
considered to be functionally neutral, offering no fitness advantage
to their hosts. They are selfish geneitc elements that exist based on
their ability to propogate themselves using the site-specific
insertion ability of the HEG without affecting the gene in which they
insert due to the self-splicing ability of the associated group I
introns.

Goddard and Burt \citep{Goddard99} offer a cyclical model of gain and
loss that can explain the prevalence and sporadic distribution of HEGs and
the group I introns in which they are found. The model has three
possible states that a possible site for a HEG+intron can be in: (1)
empty site; (2) HERE HERE HERE 



\bibliography{local}

\end{document}

%%%%%%%%%%%%%%%%%%%%%%%%%%%%%%%%%%%%%%%%%%%%%%%%%%%%%%%%%%%%%
%% BELOW KEPT FOR REFERENCE
%%%%%%%%%%%%%%%%%%%%%%%%%%%%%%%%%%%%%%%%%%%%%%%%%%%%%%%%%%%%%
\section{Usage} 

A CM is built from a multiple sequence alignment (or single RNA
sequence) with consensus secondary structure annotation denoting which
positions of the alignment are to be scored as single stranded and
which are to be scored as base paired. CMs assign position specific
scores for the four possible residues at single stranded positions and
the sixteen possible base pairs at paired positions, as well as
position specific scores for insertions and deletions. These scores
are log-odds scores derived from the observed counts of residues, base
pairs, insertions and deletions in the input alignment, combined with
prior information derived from structural ribosomal RNA
alignments. Construction and parameterization of CMs have been
described in more detail elsewhere
\citep{Eddy94,infguide03,Eddy02b,NawrockiEddy07}.

\textsc{infernal} is composed of several programs that are used in
combination to build models, search databases, and align putative
homologs, following four basic steps:

\begin{enumerate}
\item Build a CM from an input alignment.

The \emph{cmbuild} program takes as input a structural multiple
RNA alignment in Stockholm format \citep{infguide03} and creates a CM
file that is used by other \textsc{infernal} programs.

\item Calibrate the CM file for similarity search.

Prior to searching databases, parameters for approximate E-value
statistics for a CM should be estimated using the \emph{cmcalibrate}
program. This step is optional and computationally expensive (as shown
in Table~1), but is required to obtain E-values that estimate the
statistical significance of each hit in a database similarity
search. \emph{cmcalibrate} will also determine appropriate HMM filter
thresholds used to accelerate searches without an appreciable loss of
sensitivity, as described in more detail below.

\item Search databases for putative homologs.

The \emph{cmsearch} program takes a CM file as input and searches a
sequence file for high scoring hits to the model. The output of
\emph{cmsearch} includes an alignment of each hit in a BLAST-like
format augmented with structure annotation and (optionally) with
posterior probabilities indicating the confidence in each aligned
position.

\item Align putative homologs to the model.

\emph{cmalign} takes a CM file as input and a target sequence file
containing putative homologs and aligns the full length sequences to
the model, creating a structurally annotated multiple alignment in
Stockholm format.

\end{enumerate}

Some of these steps are unnecessary for some applications. For
example, a user that wants only to generate alignments of previously
defined homologous sequences, such as small subunit ribosomal RNA
sequences (SSU rRNA), would skip the calibration and search steps. 

For similarity search applications, where the goal is to identify new
examples of a family, it is reasonable to iterate these steps, adding
newly found homologs to the alignment and repeating the search as the
detected range of the family expands. Just as with primary sequence
profiles, the ability of CMs to detect remote homologs tends to
increase as the diversity of known sequences in the query alignment
increases.

\section{Performance}

A published benchmark (independent of our lab) found that
\textsc{infernal} and other CM based methods were the most sensitive
and specific tools for structural RNA homology search among the
several that were tested \citep{Freyhult07}.  Results on the benchmark
we use during \textsc{infernal} development have been consistent with
that conclusion \citep{NawrockiEddy07}. 

Figure~1 shows updated benchmark results comparing \textsc{infernal}
1.0 to the previous version (0.72) that was benchmarked in
\citep{Freyhult07}, and also to family-pairwise-search with BLASTN
\citep{Altschul97,Grundy98b}.  The sensitivity and specificity of
\textsc{infernal} 1.0 have greatly improved. There have been three
relevant improvements in the implementation: a biased composition
correction to the raw log-odds scores, the use of the full Inside
log-likelihood scores (summed over all alignments) in place of maximum
likelihood alignment scores, and the introduction of approximate
E-value estimates for the scores.

The benchmark dataset used in Figure~1 is an improved version of our
internal Rfam-based benchmark \citep{NawrockiEddy07}. Briefly, this
benchmark was constructed as follows. The sequences of the seed
alignments of 503 Rfam (release 7) families were single linkage
clustered by pairwise sequence identity, and separated into two
clusters such that no sequence in one cluster is more than 60\%
identical to any sequence in the other. The larger of the two clusters
was assigned as the query (preserving their original Rfam alignment
and structure annotation), and the sequences in the smaller cluster
were assigned as true positives in a test set. We required a minimum
of five sequences in the query alignment. 51 Rfam families with 450
test sequences met these criteria, as described in
\citep{NawrockiEddy07}. We then embed the 450 test sequences at random
positions in a 1 Mb ``pseudogenome''. Previously we generated the
pseudogenome sequence from a uniform residue frequency distribution
\citep{NawrockiEddy07}. Because base composition biases in the target
sequence database cause the most serious problems in separating
significant CM hits from noise, we improved the realism of the
benchmark by generating the pseudogenome sequence from a 15-state
fully connected hidden Markov model (HMM) trained by Baum-Welch
expectation maximization \citep{Durbin98} on genome sequence data from
a wide variety of species. Each of the 51 query alignments is used to
build a CM and search the pseudogenome, a single list of all hits for
all families is collected and ranked, and true and false hits were
defined (as described in \citep{NawrockiEddy07}) and used to build the
ROC curves in Figure~1.

CM search algorithms are computationally expensive, and
\textsc{infernal} searches require a large amount of compute time.  To
alleviate this, \textsc{infernal} 1.0 implements the HMM filtering
technique described by Weinberg and Ruzzo \citep{WeinbergRuzzo06}. By
default, the \emph{cmcalibrate} step will attempt to configure HMM
filtering thresholds to enable this acceleration (occasionally a model
with little primary sequence conservation cannot be usefully
accelerated by a primary sequence based filter).  The benchmark in
Figure~1 shows results with and without the HMM filters. The default
filters accelerate similarity search for the benchmark by about
13-fold overall, while sacrificing a small amount of sensitivity. This
makes \textsc{infernal} 1.0 substantially faster than 0.72, although
\textsc{BLAST} is still orders of magnitude faster than
\textsc{infernal}. Table~1 shows specific examples of running times
for similarity searches for six RNA families of various sizes with and
without filters. The range of speedups from filters vary widely. In
general, the more primary sequence conservation in a family, the more
effective an HMM filter will be, although this is not always the
case. Further acceleration remains a major goal of \textsc{infernal}
development.

The computational cost of CM alignment with the \emph{cmalign} program
has been a limitation of previous versions of
\textsc{infernal}. Version 1.0 now uses a constrained dynamic
programming approach first developed by Michael Brown \citep{Brown00}
that uses sequence specific bands derived from a first-pass HMM
alignment. This technique offers a dramatic speedup relative to
unconstrained alignment, especially for large RNAs such as small and
large subunit (SSU and LSU) ribosomal RNAs, which can now be aligned
in roughly 1 and 3 seconds per sequence, respectively (Table 1), as
opposed to 12 minutes and three hours in previous versions (data not
shown). We expect this to be particularly useful in applications where
many large RNA sequences need to be aligned. One of the main ribosomal
RNA databases, RDP, has recently adopted \textsc{infernal} in its
pipeline \citep{Cole09}.


\section{Discussion}

The \emph{cmbuild} program requires as input a structurally annotated
multiple sequence alignment. The \textsc{infernal} implementation
currently does not attempt to predict the consensus structure of a
sequence alignment, nor does it infer an alignment from unaligned
sequences \emph{de novo}. It is designed for (and most useful for) the
seed profile strategies used by databases such as Pfam and Rfam
\citep{Finn08,Gardner09}, in which a stable, representative,
well-annotated ``seed'' alignment of a sequence family is curated, and
a computational profile of that seed alignment (either a
\textsc{hmmer} profile HMM in the case of Pfam, or an
\textsc{infernal} CM in the case of Rfam) is used to identify and
align additional members of the family.

\textsc{infernal} remains computationally expensive. It generally
requires the use of a cluster, rather than a single desktop computer,
for most problems of interest. The most expensive programs
(\emph{cmcalibrate}, \emph{cmsearch}, and \emph{cmalign}) are
implemented in coarse-grained parallel MPI versions for use on
clusters. 

The complete \textsc{infernal} version 1.0 software package, including
documentation and ANSI C source code, may be downloaded from
\url{http://infernal.janelia.org}. It uses a GNU configure system and
should be portable to any POSIX-compliant operating system, including
Linux and Mac OS/X. It is freely licensed under the GNU General Public
License, version 3.

\section{Acknowledgements}

\textsc{infernal} development is supported by Howard Hughes Medical
Institute. It has also been supported in the past by an NIH NHGRI
Institutional Training Grant in Genomic Science (T32-HG00045) to EPN,
an NSF Graduate Fellowship to DLK, and by NIH R01-HG01363 and a
generous endowment from Alvin Goldfarb. We thank Goran Ceric for his
peerless skill in managing Janelia Farm's high performance computing
resources.

\bibliography{master,books,lab,new}

\newpage
\begin{figure}
\begin{center}
\includegraphics[width=6.4in]{figs/roc}
\caption{\textbf{ROC curves for the benchmark.}  Plots are shown for
the new \textsc{infernal} 1.0 with and without filters, for the old 
\textsc{infernal} 0.72, and for
family-pairwise-searches (FPS) with \textsc{blastn}.}
\label{fig:roc}
\end{center}
\end{figure}

\newpage

%%%%%%%%%%%%%%%%%%%%%%%%%%%%%%%%%%%%%%%%%%%%%%%%%%%%%%%%%%%%%%%%%%%%%%%
% The 6RNAs table, running times for various applications for 
% 6 RNAs
%\normalfont\ttfamily
\begin{table}
\begin{center}
\begin{tabular}{lrr|r|rr|r|} 
       & consensus & average \% id   & calibration  & \multicolumn{2}{c|}{search (min/Mb)} & alignment \\
family & length    & in training aln & (hours/model)& no filter & w/filters & (sec/seq) \\ \hline
tRNA    & 71       & 44\%            &       2.8h   &     47.3m &       8.7m&  0.013s \\
5S rRNA & 119      & 56\%            &       2.6h   &     58.6m &      10.1m&  0.026s \\
SRP RNA & 304      & 46\%            &      16.3h   &    333.7m &       5.6m&  0.168s \\
RNaseP  & 365      & 65\%            &      29.9h   &    408.9m &       1.7m&  0.176s \\
SSU rRNA& 1545     & 77\%            &      62.2h   &      n.d. &       n.d.&  1.087s \\
LSU rRNA& 2898     & 82\%            &     134.0h   &      n.d. &       n.d.&  3.072s \\
\end{tabular}
%
% alignment times (a subset of these will be in final table)
% 
% Timings from ~/notebook/8_0909_manuscript_inf-1_appnote/00LOG
% 
% family          & non-banded CYK & HMM banded CYK & HMM banded optacc (default)
%
% tRNA            &       0.049    &         0.0045 &            0.0129
% 5S rRNA         &       0.2023   &         0.0095 &            0.0256
% SRP RNA         &       5.4509   &         0.0615 &            0.1680
% RNase P         &      11.958    &         0.0721 &            0.1759
% SSU rRNA        &     724.201    &         0.7691 &            1.0870
% LSU rRNA        &   11819.9      &         2.5347 &            3.1206
%
%
% search times (a subset of these will be in final table)
% 
% Timings from ~/notebook/8_0909_manuscript_inf-1_appnote/00LOG
% times are per Mb (for forward AND reverse strand) on two input
% seq files: 10Mb.eq.fa   (25% A,C,G,U) = eq
% seq files: 10Mb.real.fa               = reall
% 
%                 ---------------------------------------------------------------------
% family          &    viterbi   &    forward    &     inside       &     filter      &
%                 ---------------------------------------------------------------------
%                 &   eq  & real & eq    & real  &  eq     &  real  &  eq    & real   & 
%                 ---------------------------------------------------------------------
% tRNA            &  6.8s & 6.8s & 20.5s & 20.5s &  2817.2s&  2840.7& 517.8s & 522.1s & 
% 5S rRNA         & 10.1s &10.0s & 32.1s & 32.0s &  3516.3 &  3517.5& 612.9s & 607.5s & 
% SRP RNA         & 22.6s &22.6s & 77.1s & 77.2s & 20099.3 & 20024.0& 878.9  & 335.3s &
% RNase P         & 26.9s &26.9s & 94.1s & 94.2s & 24533.3 & 24559.9& 124.1  & 103.3s &
% SSU rRNA        & 
% LSU rRNA        & 
%
\end{center}
\caption{\textbf{Calibration, search, and alignment running times for
    six known structural RNAs of various sizes.} SSU rRNA and LSU rRNA
    search times were not determined (n.d.); much faster non-CM
    methods are well suited for finding these RNAs, due to their high
    level of primary sequence conservation. CPU times are measured for
    single execution threads on otherwise unloaded 3.0 GHz Intel Xeon
    processors with 8 GB RAM, running Red Hat AS4 Linux operating
    systems.}
\label{tbl:times}
\end{table}




%%%%%%%%%%%%%%%%%%%%%%%%%%%%%%%%%%%%%%%%%%%%%%%%%%%%%%%%%%%
%%%%%%%%%%%%%%%%%%%%%%%%%%%%%%%%%%%%%%%%%%%%%%%%%%%%%%%%%%%
%%%%%%%%%%%%%%%%%%%%%%%%%%%%%%%%%%%%%%%%%%%%%%%%%%%%%%%%%%%

%%%%%%%%%%%%%%%%%%%%%%%%%%%%%%%%%%%%%%%%%%%%%%%%%%%%%%%%%%%
% OTHER POSSIBLE TABLES/FIGURES
%
% kept here for convenience in case they're needed later
%%%%%%%%%%%%%%%%%%%%%%%%%%%%%%%%%%%%%%%%%%%%%%%%%%%%%%%%%%%

%%%%%%%%%%%%%%%%%%%%%%%%%%%%%%%%%%%%%%%%%%%%%%%%%%%%%%%%%%%%%%%%%%%%%%%
% GC content figure 
%
\begin{comment}
\begin{figure}
\begin{center}
\includegraphics[width=6.4in]{figs/gc}
\caption{\textbf{GC content of 100 nucleotide windows in the
    benchmark pseudogenome and in a sampling of real genomic segments.}}  
\label{fig:gc-realmark}
\end{center}
\end{figure}
\end{comment}
%%%%%%%%%%%%%%%%%%%%%%%%%%%%%%%%%%%%%%%%%%%%%%%%%%%%%%%%%%%%%%%%%%%%%%%
% 6 RNAs table, different formatted
\begin{comment}
\begin{table}[htb]
\begin{center}
\begin{tabular}{ccccccr} 

       & consensus & \multicolumn{3}{c}{homology search} &  & alignment \\ \hline
family & length    & hmm & cm    & filtered & calibration   & sec/seq   \\ \hline
%                    fwd   inside non-banded
tRNA    & 71       & 20.5s & 3030s  & 522.1s   & 2.8h        & 0.013 \\
5S rRNA & 119      & 32.1s & 2941s  & 607.5s   & 2.6h        & 0.026 \\
SRP RNA & 304      & 77.2s & 25000s & 335.3s   & 16.3h       & 0.168 \\
RNaseP  & 365      & 94.1s & 50000s & 103.3s   & 29.9h       & 0.176 \\
SSU rRNA& 1545     &  ?    & ?      & ?        & 62.2h       & 1.087 \\
LSU rRNA& 2898     &  ?    & ?      & ?        & 134.0h      & 3.072 \\
\end{tabular}
\end{comment}
%%%%%%%%%%%%%%%%%%%%%%%%%%%%%%%%%%%%%%%%%%%%%%%%%%%%%%%%%%%%%%%%%%%%%%
% The rmark MER table - summary MERs
%\normalfont\ttfamily
\begin{comment}
\begin{table}[htb]
\begin{center}

\begin{tabular}{lccr} 

program & MER & hours \\ \hline
\textsc{blastn}                    & 286 &   0.02 \\
\textsc{infernal} 0.55             & 234 & 564.9  \\
\textsc{infernal} 0.72             & 185 &  84.0  \\
\textsc{infernal} 1.0 non-filtered & 109 & 159.0  \\
\textsc{infernal} 1.0 default      & 124 &  12.6  \\

\begin{tabular}{lccr} 

        & \multicolumn{2}{c}{MER} & \\
program & non-biased & biased & hours \\ \hline

\textsc{blastn}                    & 216 & 286 &   0.02 \\
\textsc{infernal} 0.55             & 232 & 234 & 564.9  \\
\textsc{infernal} 0.72             & 114 & 185 &  84.0  \\
\textsc{infernal} 1.0 non-filtered & 109 & 109 & 159.0  \\
\textsc{infernal} 1.0 default      & 120 & 124 &  12.6  \\

\end{tabular}
\end{center}
\caption{\textbf{Rfam benchmark MER summary statistics.}} 
\label{tbl:rmarkmerlist}
\end{table}
\end{comment}
%%%%%%%%%%%%%%%%%%%%%%%%%%%%%%%%%%%%%%%%%%%%%%%%%%%%%%%%%%%%%%%%%%%%%%%




